%%%%%%%%%%%%%%%%%%%%%%%%%%%%%%%%%%%%%%%%%
% Short Sectioned Assignment
% LaTeX Template
% Version 1.0 (5/5/12)
%
% This template has been downloaded from:
% http://www.LaTeXTemplates.com
%
% Original author:
% Frits Wenneker (http://www.howtotex.com)
%
% License:
% CC BY-NC-SA 3.0 (http://creativecommons.org/licenses/by-nc-sa/3.0/)
%
%%%%%%%%%%%%%%%%%%%%%%%%%%%%%%%%%%%%%%%%%

%----------------------------------------------------------------------------------------
%	PACKAGES AND OTHER DOCUMENT CONFIGURATIONS
%----------------------------------------------------------------------------------------

\documentclass[paper=a4, fontsize=11pt]{scrartcl} % A4 paper and 11pt font size

\usepackage[T1]{fontenc} % Use 8-bit encoding that has 256 glyphs
\usepackage{fourier} % Use the Adobe Utopia font for the document - comment this line to return to the LaTeX default
\usepackage[english]{babel} % English language/hyphenation
\usepackage{amsmath,amsfonts,amsthm} % Math packages
\usepackage{color}
\usepackage{sectsty} % Allows customizing section commands
\allsectionsfont{\centering \normalfont\scshape} % Make all sections centered, the default font and small caps

\usepackage{fancyhdr} % Custom headers and footers
\pagestyle{fancyplain} % Makes all pages in the document conform to the custom headers and footers
\fancyhead{} % No page header - if you want one, create it in the same way as the footers below
\fancyfoot[L]{} % Empty left footer
\fancyfoot[C]{} % Empty center footer
\fancyfoot[R]{\thepage} % Page numbering for right footer
\renewcommand{\headrulewidth}{0pt} % Remove header underlines
\renewcommand{\footrulewidth}{0pt} % Remove footer underlines
\setlength{\headheight}{13.6pt} % Customize the height of the header

\numberwithin{equation}{section} % Number equations within sections (i.e. 1.1, 1.2, 2.1, 2.2 instead of 1, 2, 3, 4)
\numberwithin{figure}{section} % Number figures within sections (i.e. 1.1, 1.2, 2.1, 2.2 instead of 1, 2, 3, 4)
\numberwithin{table}{section} % Number tables within sections (i.e. 1.1, 1.2, 2.1, 2.2 instead of 1, 2, 3, 4)

\setlength\parindent{0pt} % Removes all indentation from paragraphs - comment this line for an assignment with lots of text

%----------------------------------------------------------------------------------------
%	TITLE SECTION
%----------------------------------------------------------------------------------------

\newcommand{\horrule}[1]{\rule{\linewidth}{#1}} % Create horizontal rule command with 1 argument of height

\title{	
\normalfont \normalsize 
\textsc{Max-Planck-Institute, Stuttgart} \\ [25pt] % Your university, school and/or department name(s)
\horrule{0.5pt} \\[0.4cm] % Thin top horizontal rule
\huge Chasing $d$-wave superconductivity \\
in the $2D$ Hubbard Model % The assignment title
\horrule{2pt} \\[0.5cm] % Thick bottom horizontal rule
}

\author{Ciro Taranto, Demetrio Vilardi} % Your name

\date{\normalsize\today} % Today's date or a custom date

\begin{document}

\maketitle % Print the title

%----------------------------------------------------------------------------------------
%	PROBLEM 1
%----------------------------------------------------------------------------------------

\section{Introduction}
In this report we show our DMF$^2$RG results for the  2$d$ one band Hubbard model in the intermediate-to-strong coupling regime and with a Fermi surface structure similar to the cuprate one. 
The report is structured as follows. 
First we explain the cutoff choice that we have implemented. 
Then we show our results, namely: 
\begin{itemize}
\item With full frequency dependence vertex we manage to recover the DMFT N\' eel temperature, at weak and strong coupling; 
\item The leading instability is incommensurate antiferromagetic; 
\item At strong coupling the incommensurability vector does not necessarily correspond to the one of the particle-hole bubble; 
\item Nonlocal fluctuations only slightly decrease the DMFT-N\' eel temperature; 
\item We do not find a $d$-wave pairing instability at the temperature studied; 
\item The $d$-wave pairing fluctuations can become large close to the antiferromagnetic instability, showing that the spin-fermion mechanism is active also at strong coupling. 
\end{itemize} 
In the second part of this report, we discuss several ideas and proposals that aim to a better understanding of the data obtained thus far. 
%------------------------------------------------

\section{Method and cutoff choice}
We used a \textit{local conserving cutoff}, which means that the $\Lambda$ dependence of the Green's function is chosen in such a way that the DMFT self-consistency condition is verified for every $\Lambda$-value:
\begin{equation}
\int_{\mathbf{k}} G^\Lambda_{\mathbf{k},\nu}|_{\Sigma_{\mathrm{DMFT}}} = \mathcal{G}_{\nu}.   
\end{equation}
Here $\mathcal{G}_{\nu}$ is the Green's function of the Anderson Impurity model associated with the DMFT problem of the lattice system considered. The $\Lambda$-dependent Green's function is defined by: 
\begin{equation}
G_{\mathbf{k},\nu}^\Lambda = \left[i\nu+(1-\Lambda)\epsilon_{\mathbf{k}}+\mu +f^\Lambda \right]^{-1}
\end{equation}
%------------------------------------------------

\section{Results}
\subsection{Half filling at strong coupling} 
\begin{itemize}
\item Flow at strong coupling: flow of the magnetic channel+flow of the susceptibility; 
\item Extrapolation of N\' eel temperature from susceptibility (with self-energy);
\item Magnetic channel frequency plot, at strong and weak coupling; 
\end{itemize} 
%----------------------------------------------------------------------------------------
\subsection{Away from half filling}
\begin{itemize}
\item doping scan at fixed temperature $\beta=50$: Critical scale and pairing fluctuations; 
\item flow of the d-wave channel (at different temperatures?); 
\item Perpendicular ladder, comparison of the $d$-wave in different schemes: PL, decoupled, dmf2rg; 
\end{itemize}  

%----------------------------------------------------------------------------------------
\section{Proposed investigations} 
\begin{itemize}
\item origin of $d$-wave superconductivity;  
\end{itemize} 
\newpage
\section{Rawdata and plots to get} 
{\color{red} 
\begin{itemize}
\item flow of susceptibility+flow of the magetic (strong coupling HF); \textbf{Demetrio}
\item Extrapolation of N\' eel temperature from susceptibility (with self-energy);\textbf{Ciro}  
\item Magnetic channel frequency plot, at strong and weak coupling; \textbf{Ciro+Demetrio script} 
\item doping scan at fixed temperature $\beta=50$: Critical scale and pairing fluctuations; \textbf{Ciro} 
\item flow of the d-wave channel (at different temperatures?); \textbf{Demetrio}
\item Perpendicular ladder$\rightarrow$ comparison of the $d$-wave in different schemes: PL, decoupled, dmf2rg (function of $\Lambda$ also for the Magnetic) \textbf{Demetrio};
\item Maximum of $d$-wave as function of the maximum of $\mathcal{M}$;   
\item Decoupled vs non decoupled (AF+$d$-wave fluct) Phase diagram   \textbf{Ciro} - plot of the maximum of the $d$-wave as function of the maximum of the magnetic in two different ways. $a)$ fixed doping, $b$ various dopings (final $\Lambda$) ;
\end{itemize} 
 } 
 
 \section{Ideas and proposal (easy or in progress)}  
 
 \begin{itemize}
 \item Fermi Surface at strong coupling: Is the Fermi surface prone to have hot spots, also in the presence of a relatively large self-energy? How is this important for the superconductivity. 
 \end{itemize}
 \section{Further studies (hard or to be planned)}  
\begin{itemize}
\item $d$-wave fluctuations beyond the critical scale: how to get them?  
\end{itemize} 
\end{document}